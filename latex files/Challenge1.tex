\documentclass{article}

\usepackage{fancyhdr}
\pagestyle{fancy}
\lhead{Ruicheng Wu}
\rhead{06/30/2017}
\chead{Challenge problem}

\begin{document}
1.\\$m/n$ can have remainder $0,1,2...n-1$,there will be infinite remainders (even $1/1$ can be kept expressing as $1/1$),so according to pigeon hole principle, there will be repeating remainder after dividing by $n_th$ number.So number of places of $n$ is the upper bound . 

2.\\Construct a model of this as a graph. The string leading to a person stands for the degree of a node in graph. So,the sum of
degree of every node is actually twice of the total edge number,which is the actual number of strings. Because every string is counted twice. Given the total number of degree $S$(people can get this by communicating with each other),by division principle, the total string number is $S/2$.

3.\\Obviously,$C(a+b,b)$ is the number of total ways to construct a monotonic path by thinking there is $a$ ways to go horizontally, $b$ ways to go vertically. In order to find the ways that go stay or below $y=x$,by subtraction rule, the total ways minus the ways that goes above $y=x$ is the answer.
$C(a+b,b-1)$ when $b \leq a$ stands for given a total arrangement of $HHHHHHVVVVVV$ with $a H's, b V's$ that choose $b-1$ spots to put $V's$, for the remaining one $V$, there is always one situation that makes the path go above the line. So choosing paths going above is $C(a+b,b-1)$. To choose path stay or below is $$C(a+b,b) - C(a+b,b-1)$$ 

4. \\According to figure 1, a $4x4$ grid with a corner removed can be constructed from this tile. Since the given range of grid is $2^n x 2^n$ with $n>=2$ it can always be constructed by $4x4$ grid. In order to do this, just keep split the entire gird by 4 until each unit is a small $4x4$ grid. The one with a corner removing can be immediately tiled by figure 1, the rest 3 actually an spare another $'L'$ shape when their missing corners are next to each other, then just put another small unit tile will construct a entire $8 x 8$ square. By keep doing so, any $2^n x 2^n$ square can be constructed.

5. \\Multiset permutation : $$15!/5!5!5!$$
Size of identical necklaces with rotation and reflection : $$15 * 2$$
Result : $$15!/(5!5!5!(15*2))$$

6.Everything is based on whether a set contain duplicate or not. Say $${(1,1,2,3}) == ({1,2,3}) = [3]$$, $n = 3$. Then his method won't work. Assume given $$(1,1,2,3) , n_1 = 2,n_2 = 1,n_3 = 1$$,Obviously there is $3$ ways to choose size 2 set contains 1: $(1,1),(1,2),(1,3)$,but his formula gives $C(3-1,2-1) = 2$. If the given set doesn't contain any duplicate ,say $(1,2,3)$,then his formula gives same result as counting. So I claim that Count Chocula's method only works for set that doesn't contain duplicate. By giving this, first part contains 1 will have 1 choose already,so the range number to choose is n-1, then there is only $n_1-1$ ways to choose other element. And so forth. The things matter here is n,if n stands for the total numbers of element,this formula will work; if n is just the max number of set [n], it won't work.
	 
7.\\Think of a example for 5;$$(5), (4, 1), (3, 2), (3, 1, 1), (2, 2, 1), (2, 1, 1, 1), and (1, 1, 1, 1, 1)$$
Now function: $$p(n,k) =$$ number of partitions of n with at most k parts
$$q(n,k) = $$ number of partitions in which each part is of size at most k 
here.Giving an example, $$p(5,3) = |{(5), (4, 1), (3, 2), (3, 1, 1), (2, 2, 1)}|= 5 $$
and $$q(5,3) = |{(3, 2), (3, 1, 1), (2, 2, 1), (2, 1, 1, 1),(1, 1, 1, 1, 1)}| = 5$$
I observe that one element from $p(n,k)$ is corresponding to another value in $q(n,k)$. In this case, $(5) -> (1,1,1,1,1)$;
$(4, 1) -> (2, 2, 1)$. Because the total sum of both sides are same, I will construct a graph made by $1x1$ unit cubes. Think $(5)$ as 5 cubes lining up in one row, while $(1,1,1,1,1)$ as 5 cubes lining up in one column. The same thing apples to $(4, 1) -> (2, 2, 1)$. 

Assume an element from $p(n,k)$ is constructed by graph $(p_1,p_2...p_k)$,then its mapping element in $q(n,k)$ is constructed by its symmetrical graph by all columns switching with rows.

Since for every element in $q(n,k)$ can and only can be found in one corresponding element in  $p(n,k)$, it is one to one relationship between this two functions.\\So it proves number of partitions of n with at most k parts = number of partitions in which each part is of size at most k 

 
\end{document}
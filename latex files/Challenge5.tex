\documentclass{article}

\usepackage{fancyhdr}
\pagestyle{fancy}
\lhead{Ruicheng Wu}
\rhead{07/30/2017}
\chead{Challenge Problem 5}



\begin{document}
3.

We can form a bipartite graph:

Set 1 : All n rows of this latin squares

Set 2 : All n symbols

Edges : connect each symbol to each row if and only if such symbol $a$ does not show up in row $b$.

This graph satisfies Hall’s condition.Consequently, it has a perfect matching.This perfect matching stands for pair each row with a different
symbol  up, with each paired symbol does not occur in that row yet. 

Hence this perfect matching can produce 1 more columns for currently m x n matrix. If we repeat this step will eventually ends up with n x n latin square.

4.

First we can choose a random row or column ,which has 2n choices. 

Let X be the number of distinct entries in it. Now X = $\sum K_i$, where each $K_i$ is the token of i appearing (may be more than one time)in the row/column that we choose. 

Obviously, each $E[K_i] = P [K_i \geq 1]*n$. In order to solve this problem, we will find the lower bound.

By observing that the worst-case is for each i,if its all n appearing times are in some $\sqrt[]{n} * \sqrt[]{n}$ matrix,which turns out $P [K_i \geq 1] \geq \frac{2\sqrt[]{n}}{2n} = \frac{1}{\sqrt[]{n}}$. 

Hence,this is a linear probabilistic model, the total number is n. $$E(X) \geq P [K_i \geq 1] * n = \sqrt{n}$$

5.

We are going to solve this problem by setting a model using probabilistic method.

Think of flipping a fair coin for each node; if it is head, we put the node in a set S and if it is tail, we do nothing.

So we have a partition of the nodes into two sets, S and $V \S$. Now consider the subgraph obtained by just taking the edges crossing this partition, then we have created a bipartite subgraph of G.

So the expected number of edges in the subgraph (in the cut) is $\frac{|E|}{2}$(because every edge has a 1/2 chance to be the crossing edge).

By this probabilistic method, there must exist some subgraph with at least $\frac{|E|}{2}$ edges.

6.

We first connect the these 5 points.If there is a pentagon ,it is guaranteed to have a convex quadrilateral(just removing one point ant its edges). No quadrilateral by 5 points since no 3 points on a line.

Here is another situation,we can obtain a triangle making by 3 points $(J,K,L)$ that the rest 2 points $A,B$ are inside of it.

$A,B$ can make a straight line that intersects with the triangle at two sides.
This is because no 3 points can be in one line.So it is impossible for this line to go through one point of the triangle then intersect only one side. There must be 2 sides to intersect.

We then just find the lines from triangle cross by two points $J,K$ which does not intersect with $AB$ (in the triangle only!).

J,K,A,B all together can make up a convex quadrilateral

7.

We think this expectation as all $B_i$ be 1,
this explains $2^{-n}$ on the RHS.

Then we can just apply the proof of Sperner’s Lemma.

$$Pr[|\sum_{i=1}^n X_i|] \leq C(n,floor(\frac{n}{2}))$$

By linear probabilistic method, the entire summation 

\begin{huge}
$$Pr[|\sum_{i=1}^n X_i* B_i|] \leq C(n,floor(\frac{n}{2}))* 1 * 2^{-n}$$
\end{huge}

There is an antichain X in $B_n$. Then
$x_i = $ \{ $|X|=i$ \} will satisfy this equation.

8.

a)

It is going to be the diagonal line of the matrix,specifically,$B = M^T*M$. It is just sum of square of each element in the i-th row. And we use 1 to stand for those i-th edge contains the j-th vertex. It is just like a laplacian of this matrix.

$$B_{ij}={|V_1|,|V_2|,|V_3|...|V_m|;m is the edge number}$$

It is the vertices that i-th edge covers.

b)

from a), we can know $B_{ij},i != j$ stands for how many numbers of vertices that i-th edge intersect with other edges.

in the example

$B_{ij} = (2,2,2,2,2,2,2,2,2..2)$,since every edge contains two comman vertices

c)

It is obviously that we can know if one vertex is covered by at least one edge. In B, on the diagonal,that edge will have at least 1 in the spot.

If no other edges share this vertex, the other spots on that row will be 0 in B.Hence according to the definition of rank of a matrix. rkB is actually depending on the number of column, n .

Else if there are other edges share that vertex, fill those spot with corresponding value. Just to remember the diagonal will certainly have a different bigger value than other values in same row,so there is no more echelon form to conduct also. It is depending on the n too.

d)

Recall $rk(XY) \leq rk(Y)$
 
$rk(M^T*M) \leq rk(M^T)$

We know $M^T$ is a n x m matrix.From c), we
know $rk(M^T)$ = m.

Hence, $rk(M^T*M) \leq m$

$$rk(B) \leq m$$

e)

Can i say this is so obvious?

because we have $rk B \leq m and rk B = n$

problem proved..

$$n \leq m$$




\end{document}
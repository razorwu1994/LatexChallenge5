\documentclass{article}

\usepackage{fancyhdr}
\pagestyle{fancy}
\lhead{Ruicheng Wu}
\rhead{06/27/2017}
\chead{Workshop 0 question 4}


\begin{document}
Prove that, for any $n + 1$ integers $a_1$, $a_2$, ..., $a_n+1$, there exist two of the integers $a_i$
and $a_j$ with $i != j$ such that $a_i − a_j$ is divisible by $n$.
\bigskip  


At first,if there exist at least two numbers $a_i$ and $a_j$ and they are equal to each other. Then their difference 0 are always divisible by n.

Now we proceed to another situation that no two numbers equal to each other.

When we want to find out if a number is divisible by n,the remainder can be any integer from $0$ to $n-1$. Totally,that is $n$ numbers.

According to the pigeon hole theory.Think each remainder as a container, each number as a pigeon. Given $n+1$ numbers,there must be $2$ numbers divide by $n$ that result in same remainder.

Let's assume those two numbers $a_i$ and $a_j$ are:

$a_i = an + r$
			
$a_j = bn + r$

When we take the absolute difference(Here I assume $a_i > a_j$):
$a_i - a_j = (a-b)n$
Their difference has a factor $n$.That proves this question.

\end{document}
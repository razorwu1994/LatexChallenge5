\documentclass{article}

\usepackage{fancyhdr}
\pagestyle{fancy}
\lhead{Ruicheng Wu}
\rhead{07/23/2017}
\chead{Challenge Problem 4}


\begin{document}

1.

Given there is a vertex with maximum degree d in a connected graph, we know that there are $d+1$ vertices. 

Minimum case:(one vertex with degree d-1)
Assume there is only one vertex with degree $d-1$. It means now we have a vertex $v_1$ connecting to all other vertices and also  a vertex $v_2$ connecting to all other vertices except one $v_{d+1}$.

Given d colors, if we want to do proper color on this graph. We will know $v_1$ and $v_2$ must be different color since they are adjacent to each other. The rest vertices $v_3,v_4...v_d$ must be different from these two colors since each of them are at least adjacent to $v_1$ and $v_2$.So they can be distributed with maximum $d-2$ colors. Now consider $v_{d+1}$,it is not adjacent to $v_2$, it could take any colors except the one $v_1$ use. 

In total,we can use $2+d-2 = d$ colors to color the graph.

Maximum case(up to d vertices with degree d-1)
This case, we will have one vertex $v_1$ with $d$ degree. Among other vertices, every one vertex will be disconnected with only one vertex because every one of them are $d-1$ degree(But they must connect $v_1$). So that means $v_2$ may not be adjacent to $v_3$,and vice versa;$v_4$ may not be adjacent to $v_5$...And so forth.Since each vertex will have another one non-adjacent to it.We can still color two by two with d-2 colors for $v_2,v_3...v_{d-1}$($v_1$ has 1 color already).Now we may end up with just one vertex or two vertices(depends on if d is odd or even), simply we can just use 1 color to color them since those two are non-adjacent to each other. Thus $1+d-2+1 = d$ colors can be used to proper color this graph.

Since we prove both the least case and the max case will satisfy this condition, we prove the whole thing.

2.
(No matter how this formula looks nonsense to me I will just prove it)

Since it is a tree : 

It will have n-1 edges;

There are n vertices since the order n of tree defines that.

I will conduct two equations:

\begin{large}

$$EqOne = d_1+d_2...+d_{n-1} = n$$

no way there will be vertex with n degree!

$$EqTwo = 1*d_1+2*d_2...+(n-1)*d_{n-1} =2|E| = 2(n-1)=2n-2$$

\end{large}

EqTwo-EqOne=

$$d_2+2*d_3+3*d_4...+(n-2)d_{n-1}=n-2$$

LHS can be separate as:

$$(d_2+d_3+d_4...+d_{n-1})+(d_3+2*d_4+3*d_5...+(n-3)*d_{n-1})$$

So the equation now would be :
$$(d_3+2*d_4+3*d_5...+(n-3)*d_{n-1}) = n-2 - (d_2+d_3+d_4...+d_{n-1})$$

=$$2+d_3+2*d_4+3*d_5...+(n-3)*d_{n-1} = n-(d_2+d_3+d_4...+d_{n-1}) = d_1$$

which proves the question.

3.





5.

Treat De Bruijn sequence as connected directed graph like $DB_n$.
We note that a De Bruijn sequence of order n is just an Eulerian
circuit of $DB_n$. 
First of all, we need to prove that an Eulerian circuit exists. According to the theorem 0.1 : "A looped directed multigraph
G with no isolated vertices has an Eulerian circuit if and only if the in-degree and out-degree of every vertex is equal to each other."

Given a string $S_0S_1...S_{k-2}$ of length $n > 1$ written with k symbols from $S_k={a,b,c...}$, there are k strings that could follow it in a De Bruijn sequence:
(Their length are n)

$S_0S_1...S_{k-2}S_0$, 

$S_0S_1...S_{k-2}S_1$,

$S_0S_1...S_{k-2}S_{k-1}$,

...

We can create a de Bruijn graph - with $k^n$ vertices, labelled by the $k^n$ strings of length n in k symbols. We join the node labelled $S_0S_1...S_{k-2}$ to other nodes with labelled edge which the node "to" begins. For example, the edge joining $S_0S_1...S_{k-2}S_0$ to $S_1...S_{k-2}S_0S_1$ is labelled "$S_1$".

Similary, there are also k strings that could have preceeded it :
(Their length are n)

$S_0S_0S_1...S_{k-2}$, 

$S_1S_0S_1...S_{k-2}$, 

$S_{k-1}S_0S_1...S_{k-2}$,

...

This tells us that every vertex on a de Bruijn's graph has both indegree and outdegree equal to k.

Additionally, a DB graph is necessarily connected. There is a path from any vertex $S_0S_0S_1...S_{k-2}$ to any other vertex $S_0S_1...S_{k-2}S_{k-1}$. For example. For n = 2, here is a path from $S_0S_2$ to $S_1S_4$:

$$S_0S_2, S_2S_1, S_1S_4 , S_4S_0 , S_0S_2$$(A cycle here)

It follows that a DB graph always has an Euler circuit. Pick any such circuit, record the successive edge labels in a string. The result will be one of de-Bruijn sequence.


Week 1 fix:

no fix :-p(would take up more times than i have to rethink these problems...)





\end{document}
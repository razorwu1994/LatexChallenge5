\documentclass{article}
\usepackage{graphicx}
\usepackage{fancyhdr}
\pagestyle{fancy}
\lhead{Ruicheng Wu}
\rhead{07/16/2017}
\chead{Challenge 3}

\begin{document}

1.

Given k standard dice,$(x+x^2...+x^6)$ is the generating function for each dice;
So the whole generating function is h(n) = $(x(1+x...+x^5))^k$,we are looking for the coefficient of $x^n$.

 $$(x(1+x...+x^5))^k= x^k(\frac{1-x^6}{1-x})^k$$
 $$=x^k*(1-x^6)^k*(1-x)^{-k}$$
 $$=x^k\sum_{l=0}^k(-1)^l*C(k,l)*x^{6l}*\sum_{m=0}C(k+m-1,m)x^m$$
 
We can see in order to make $n = k + 6*l + m$,the coefficient is actually:
$$\sum_{l=0}^k(-1)^l*C(k,l)*C(n-6l-1,n-k-6l)$$

From $C(n-6l-1,n-k-6l)$, $ n-k-6l > 0$ only when $n > k+6l$;it is not better than $C(n-6l-1,k-1)$

The answer:the coefficient of $x^n$ is $$\sum_{l=0}^k(-1)^l*C(k,l)*C(n-6l-1,k-1)$$

2.

By finding initial cases:

$a_0=1$

$a_1=1$

$a_2=2$

$a_3 =5 = 2*2+1^2$

$a_4=14=5*2+2^2$

$a_5=37=14*2+3^2$

$a_n=2*a_{n-1}+(n-2)^2$

Treat this whole sequence as combination with "+"'s and "-"'s. The rule is :
there are n "+"'s and n "-"'s.

Also for last two characters: "$-$ $-$" or "$+$ $-$" is the only option to choose. We have "$+$ $-$",think it just as a intact material that won't affect any other sequence because the rest sequence just follow the last element rule as begining,So there is a $a_{n-1}$;"$-$ $-$",we first take it as $a_{n-1}$, now total has $2*a_{n-1}$ so far;and then there has to be "$+$ $-$" in somewhere from rest sequence to complement this "--" situation;so there are "n-2" left for both "+"'s and "-"'s,so there are total $(n-2)^2$ combinations; summing up all cases given :
$a_n=2*a_{n-1}+(n-2)^2$ for n >= 2. Since $a_1$ has only 1 "+" or "-" to play around,let's just treat them as base case.

By solving this relation:

$a_n=a_0+a_1x+a_2x^2...$

$-2xa_n=0-2a_0x-2a_1x^2...$

$-\sum_{n \geq 2}(n-2)^2x^n=0-0-(3-2)x^3...$

Adding them up:

$(1-2x)a_n=1-x+\sum_{n \geq 2}(n-2)^2x^n$

$$a_n=\frac{\sum_{n \geq 0}C(n-2,n)x^n+\sum_{n \geq 2}(n-2)^2x^n}{(1-2x)}$$


$$a_n= \sum_{n \geq 2}(C(n-2,n)+(n-2)^2)x^n*\sum_{n \geq 0}2^n*x^n$$


$$a_n=\sum_{n \geq 2}2^n*(C(n-2,n)+(n-2)^2)*x^n$$

So the coefficient is :$$2^n*(C(n-2,n)+(n-2)^2)$$

3.
For distinct part: suppose each distinct partition can only be happened or not

k stands for value of each partition. The sum of those k's is equal n.
$$(1+x)(1+x^2)*(1+x^3)*...(1+x^k)$$
$$=\frac{1-x^2}{1-x}*\frac{1-x^4}{1-x^2}...\frac{1-x^{2k}}{1-x^k}$$
$$=\frac{1}{(1-x)*(1-x^3)*(1-x^5)...(1-x^l)}$$ 
\begin{small}
note: l is odd here
\end{small}

Above equation turns out to be :
$$=(1+x+x^{1+1}+x^{1+1+1}+...)*(1+x^3+x^{3+3}+x^{3+3+3}+...)*...$$ 

The equation here is exactly the number of ways to partition $n$ into k parts as k is odd.

So the distinct part generating function can be converted into odd part generating function, which proves the question. 






\end{document}
\documentclass{article}

\usepackage{fancyhdr}
\pagestyle{fancy}
\lhead{Ruicheng Wu}
\rhead{07/9/2017}
\chead{Challenge Problem 2}


\begin{document}
1.\\We can conclude that $\phi(p^k) = p^k-p^{k-1}=p^k*(1-\frac{1}{p})$.Since p is prime,there is $p^{k-1}$ numbers can be divisible to $p^k$,like $p,2p...p^{k-1}*p$ in total.Subtracting those numbers from $p^k$ will give other numbers which are prime to $p^k$.

Since for $p_1...p_i$ are distinct prime numbers dividing n(1 doesn't count).There is always an equation existing: $n=p_1^{j_1}*p_2^{j_2}...*p_i^{j_i}$. Applying $\phi(n)$, will give $$p_1^{j_1}*(1-\frac{1}{p_1})*p_2^{j_2}*(1-\frac{1}{p_2})*...*p_i^{j_i}*(1-\frac{1}{p_i})$$

equals:$p_1^{j_1}*p_2^{j_2}*...p_i^{j_i}*(1-\frac{1}{p_1})*(1-\frac{1}{p_2})*...(1-\frac{1}{p_i})$

$=n*(1-\frac{1}{p_1})*(1-\frac{1}{p_2})*...(1-\frac{1}{p_i})$, thus proves the equation.

2.\\Given $f_n-1+f_n = f_{n+1}$

$f_0 = 0,f_1 = 1,f_2 = 1,f_3 = 2,f_4=3,f_5=5$

$f_0^2 = 0,f_1 ^2= 1,f_2 ^2= 1,f_3 ^ 2= 4,f_4 ^2=9$

We observe that $S(n) = f_0^2+f_1^2+f_2^2...+f_n^2$,$S(3) = 6 = 2*3;S(4)=15 = 3*5$

Make a assumption about the formula : $f_n*f_{n+1}=S(n),n \geq 0$,n is integer.

By induction:
Base case:$S(0)=0 = 0*1$, satisfies the case.

Now we want to prove $S(n+1)=f_{n+1}*f_{n+2}$ by giving $f_n*f_{n+1}=S(n)$.

Both sides add $f_{n+1}^2$

RHS:$S(n+1)$

LHS:$f_n*f_{n+1}+f_{n+1}^2 = f_{n+1}*(f_{n+1}+f_n)$.By the basic formula given at the beginning, $f_{n+1}*(f_{n+1}+f_n) = f_{n+1}*f_{n+2}$

RHS = LHS,now it proves the equation.

3.\\By trying small n values, there are :

$t_1 = 1,t_2 = 3,t_3 = 7,t_4 = 15$, assumption about the recurrence formula is :
$a_n=2*a_{n-1}+1,n>1$.Unfortunately,it is not a linear recurrence relationship, the guess method won't apply here.Instead of that ,we do a simple expansion:

$a_0=1;a_1 = 1;a_2 = (2*(a_1)+1);a_3= (2*(a_2)+1)=(2*((2*(a_1)+1)+1)$;

for $a_n=2*a_{n-1}+1$$$=2^{n-1}*a_1+(1+2+2^2+...2^{n-2})$$
$$=2^{n-1}+2^{n-1}-1=2^n-1$$.By plug $t_1,t_2,t_3$ in, it satisfies the equation.

By induction, we want to prove $t_{n+1} = 2^n-1$,we start with $t_n = 2^{n-1}-1$. Adding $t_n+1$ on both sides.

LHS:	$t_{n+1}$

RHS:	$(2^{n-1}-1)+(2^{n-1}-1)+1=2^n-1$,thus proves the equation.

4.\\Total cases:$6^k$

There are k standard dices(value 1,2,3,4,5,6) sum up to n.

Generating function :$f_1(x)=x+x^2...+x^6$,$f_2(x)=x+x^2...+x^6$...$f_k(x)=x+x^2...+x^6$.

According to Newton’s binomial theorem:(no restriction on dice)
$$f(x)=\sum_{k \geq 0} C({n+k-1},k)x^k$$

Suppose $a_1+a_2+...a_k = n$,$1 \leq a_k \leq 6$

By substitute $a_k=b_k+1$,then $b_1+b_2+...b_k = n-k$,$0 \leq b_k $,$$f(x)=\sum_{k \geq 0} C({n-k+k-1},k)x^k=\sum_{k \geq 0} C({n-1},k)x^k$$
However this case will include dice has value over $7$.\\
That is we want to find $c_1+c_2+...c_k = n-k$,$7 \leq c_k $, substitute $ c_k=d_k+7$,$d_1+d_2+...d_k = n-k-7k=n-8k$,thus the case over $7$ is $$\sum_{k \geq 0} C({n-7k-1},k)x^k$$.

By exclusion rule,the probability is $$\frac{\sum_{k \geq 0} C({n-1},k)x^k-\sum_{k \geq 0} C({n-7k-1},k)x^k}{6^k}$$






\end{document}
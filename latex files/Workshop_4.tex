\documentclass{article}
\usepackage{graphicx}
\usepackage{fancyhdr}
\pagestyle{fancy}
\lhead{Ruicheng Wu}
\rhead{07/22/2017}
\chead{Workshop 4}


\begin{document}
3.

Firstly,we distinguish 5 edges of the outer $C_5$, 5 edges of the inner $C_5$, and 5 crossing edges.

A Hamiltonian cycle must end at its starting point, and so must use an even
number (in this case ,2 or 4) of crossing edges.

If there are 2 crossing edges, then their ends in both the outer $C_5$ and
the inner $C_5$ must be joined by paths of length 4. This implies that these
ends must be adjacent in both the outer $C_5$ and the inner $C_5$, which
is impossible.(In the graph,the inner vertex shown in-order is not adjacent connected)

If there are 4 crossing edges, then we must have three edges in both the
outer $C_5$ and the inner $C_5$, including two through each of the points
not on chosen crossing edges. There is a unique such configuration, but it consists
of two 5-cycles rather than a Hamiltonian cycle.

By conclusion of above situations,there is no Hamiltonian cycle in a Petersen graph.




\end{document}
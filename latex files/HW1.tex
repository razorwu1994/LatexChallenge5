\documentclass{article}

\usepackage{fancyhdr}
\pagestyle{fancy}
\lhead{Ruicheng Wu}
\rhead{06/28/2017}
\chead{Homework 1}

\begin{document}
1.

Step$1$:the total is a permutation of 5 elements

Step$2$:a function $f$ can map (Rainbow,Red Ballon,Clover,Horseshoe,Heart) to {Rainbow,Red Ballon,Clover,Horseshoe,Heart}(Order matters), it is a $k$ to one relationship

Step$3$:we can tell 
$$f(Rainbow,Red Ballon,Clover,Horseshoe,Heart)$$$$=f(Heart,Rainbow,Red Ballon,Clover,Horseshoe)$$
$$=f(Red Ballon,Clover,Horseshoe,Heart,Rainbow)$$ 

with total 5 cases map to 1 same result. So it is a 5 to 1 relationship.

Step$4$:hence, according to division principle, we take $5!/5 = 4! = 24$ different combinations\\

2.

a. with repetition, it is $5^5$ sequences

b. with repetition,since first and last spot has assigned by $A$,the rest spot can have $5^3$ sequences.Thus the total is $5^3*1*1$.

c. length 0,1 won't apply to this question. 

Let's start with length 2: $CD$, 1 way to do this

length 3: choice one: $CD$ is a whole part,now there are 2 options to put $CD$; choice two: no matter where to put $CD$, the rest spot can not be $C$, so there are 4 choices. By multiplication principle, Total is $2*4=8$

length 4: choice one: treat $CD$ as a whole part, now there are 3 options; choice two: no more $C$,rest two spots can have a $4^2=16$ choices.By multiplication principle, Total is $3*16 = 48$

length 5: choice one: treat $CD$ as a whole part, now there are 4 options; choice two: no more $C$,rest three spots can have a $4^3=64$ choices.By multiplication principle, Total is $4*64=256$

Grand total is $1+8+16+64 = 89$ according to addition principle.

d. 

Choice 1: there are $(6,2)$ combinations to put those two B's

Choice 2: other 4 spots can't have any B,so it is $4^4 = 256$

Total: by multiplication principle, the grand total is $6!/(2!4!)*256 = 3840$

e. 

Split the situations by length

First of all by pigeon hole principle, the length of the sequence can be more than 5 when each letter appears at most once.

length 0: if the empty sequence counts,otherwise ignore this case

length 1: $C(5,1) = 5$ , each letter appears once

length 2: $5^2-5 = 20$ , subtraction principle, all permutation subtract case when sequence contains duplicate letters

length 3: 
choice 1 : choose 3 letters $C(5,3) = 10$
choice 2 : permutation of those 3 letters $3! = 6$
total : $10*6=60$

length 4:
choice 1 : choose 4 letters $C(5,4) = 5$
choice 2 : permutation of those 4 letters $4! = 24$
total : $5*24=120$

length 5:
choice 1 : choose 5 letters $C(5,5) = 1$
choice 2 : permutation of those 5 letters $5! = 120$
total : $1*120=120$\\

3.

I will divide this question into two parts since it is asking numbers of distinct pairs of committees.
For engineer committee: $C(13,5) = 1287$
For scientist committee: $$C(7,2)+C(7,3)+C(7,4)+C(7,5)+C(7,6)+C(7,7) = 120$$
The result will be these two possibilities multiplying each other : $1287 * 120 = 154440$\\

4.

a.
Assume there are two disjoint sets,A,B. A = ${x_1, . . . , x_m}$ and B = ${y_1, . . . , y_n}$. LHS is number of subsets of $A∪B$ of size r. RHS counts same thing by selecting k elements from A ($C(m,k)$ ways), then selecting the remaining $r − k$ elements from B ($C(n,r−k)$ ways), k is integer ranging from 0 to r,that is all the cases which equals the LHS.	Thus prove the equation.


b. 
Assume there are two disjoint sets,A,B with size n respectively. LHS counts a size 2 subset of $A∪B$. RHS first term counts a size 2 subset from each set A,B. By addition rule, $C(n,2)+C(n,2)$. Second term counts the situation when one element comes from A,one element comes from B. The grand total,add these up, counts the same thing as RHS. Thus prove the equation.




\end{document}